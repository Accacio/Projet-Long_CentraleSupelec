\mysubsection{Trajectoire}

\paragraph{1.A.1 Création d'une trajectoire linéaire entre deux points}
A partir du point initial et le point final, trouver une paramétrisation pour x, y et z de façon que le trajectoire soit linéaire.                                                 
\paragraph{1.A.2 Calcul du polynôme interpolateur} 
Calculer deux polynômes interpolateurs pour que le début e le but de la trajectoires aient la vitesse et la accélération nulle, tournant le mouvement suffisamment lisse. 

\paragraph{1.A.3 Calcul du cercle}
A partir du point initial, le point intermédiaire et le point final, trouver un cercle que passe par ces trois points

\paragraph{1.A.4 Paramétrisation du cercle}
Gérer une paramétrisation en X,Y et Z pour le cercle.  Exemples: $ X=1-cos(t) $ et $ Y=sin(t) $, ou $ X=t $ et $ Y=\pm \sqrt{1-t^2}$.

\paragraph{1.A.5 Calcul du angle du arc}
Calculer le angle formé entre le point initial, le centre du cercle et le point final, a fin de former juste un arc de cercle.

\paragraph{1.A.6 Calcul du polynôme interpolateur}
Calculer deux polynômes interpolateurs pour que le début e le but de la trajectoires aient la vitesse et la accélération nulle, tournant le mouvement suffisamment lisse. 


%\todo{Dados 3 pontos calcular circulo que passa por eles\\
%Fazer parametrização do círculo, (Seno e 1-cos)\\
%Calcular ângulo entre ponto inicial, centro do círculo e ponto final.\\
%Calcular polinômio interpolador com derivada igual a 0 quando ângulo é zero e derivada igual a cosseno e o valor é seno de algum angulo (1/9 do total)\\
%Calcular polinômio interpolador com derivada igual a cosseno e o valor é seno de algum angulo (8/9 do total) e derivada igual a 0 quando ângulo é equivalente onde queremos parar.}