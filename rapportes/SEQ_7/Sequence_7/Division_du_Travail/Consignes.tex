\mysubsection{Commande Angulaire et Cartésienne}

En réalité, on a déjà commencé à travailler à cette partie du projet dans la séquence 7, car pour développer une loi de commande pour un système, on a besoin de premièrement le modéliser. Comme le modèle du robot nous a été fourni dans un fichier Matlab, on a employé du temps en le comprendre. Alors, la tâche 1.B.1 qui est décrite dans cette section ont été réalisées par tous les membres de l'équipe, et on a déjà fini jusqu'à la tâche 1.B.4.
On a réservé 28 créneaux pour les tâches de cette structure, dont 14 sont compris dans les activités de la séquence 7, qui étaient nécessaires pour la suite de cette partie du projet.

Dans la suite, la division de cette partie du travail en tâches est décrit, ainsi que le temps qu'on a l'intention de consacrer à chaque tâche.

\begin{itemize}
\item 1.B.1: Étude à propos de la modélisation des systèmes robotiques: 6 créneaux;

\item 1.B.2: Compréhension du modèle Matlab du robot fourni: 2 créneaux

\item 1.B.3: Étude et développement des techniques de commande du robot dans l'espace des articulations et choix d'une méthode de commande: 4 créneaux;

\item 1.B.4: Mis en oeuvre de la loi de commande du robot dans l'espace des articulations dans le Simulink et tests: 2 créneaux;

\item 1.B.5: Étude et développement des techniques de commande du robot dans l'espace opérationnel et choix d'une méthode de commande: 2 créneaux;

\item 1.B.6: Mis en oeuvre de la loi de commande du robot dans l'espace opérationnel dans le Simulink et tests: 4 créneaux;

\item 1.B.7: Optimisation des lois de commande: 2 créneaux;

\item 1.B.8: Intégration avec le module de génération des trajectoires: 3 créneaux;

\item 1.B.9: Intégration avec le robot: 3 créneaux;

\end{itemize}