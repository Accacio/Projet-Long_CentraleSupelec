\usepackage{enumerate}
\usepackage{graphicx}
\graphicspath{{Figuras/}}
\usepackage{color}
\usepackage[cmex10]{amsmath}
\usepackage{array}
\usepackage{float}
\usepackage[utf8]{inputenc} 
\usepackage[french]{babel}
\usepackage[font=normalsize,format=plain,labelfont=bf,up,textfont=up,figurename=Figura,tablename=Tabela]{caption}
\usepackage{subcaption}
\usepackage[top=1in, bottom=1in, left=1.25in, right=1.25in]{geometry}
\usepackage{indentfirst}
\usepackage{fancyhdr}
% Font packages
\usepackage{amssymb}
\usepackage{amsfonts}
\usepackage{pgfgantt}
\usepackage{steinmetz}
% Nice extra font package, e.g. \mathds{1}
\usepackage{dsfont}
\usepackage{color}
\usepackage{blindtext}
% Use multiple rows when writing tables
\usepackage{multirow}
\usepackage{booktabs}
\usepackage{bm}
\usepackage{bigstrut}
% Uncomment next line to make footnots per page
\usepackage{perpage}
% Uncoment next group of lines to create the table of contents for the PDF
\usepackage{hyperref}
\definecolor{darkblue}{rgb}{0,0,0.5}
\renewcommand{\title}{Projet Long}
\newcommand{\subtitle}{Rapport d'activité de la Séquence 7}
\hypersetup{
    pdftitle={\title},
    pdfauthor={},
    bookmarksnumbered=true,     
    bookmarksopen=true,         
    bookmarksopenlevel=1,       
    colorlinks=true,
    linkcolor=darkblue,
    filecolor=darkblue,  
    urlcolor=darkblue,  
    citecolor=darkblue,              
    pdfstartview=Fit,          
    pdfpagemode=UseOutlines,    % this is the option you were lookin for
    pdfpagelayout=TwoPageRight
}
\let\oldcontentsline\contentsline%
\renewcommand\contentsline[4]{%
    \oldcontentsline{#1}{\smash{\raisebox{1em}{\hypertarget{toc#4}{}}}#2}{#3}{#4}}

\newcommand\mysection[1]{\section[#1]{\protect\hyperlink{tocsection.\thesection}{#1}}}
\newcommand\mysubsection[1]{\subsection[#1]{\protect\hyperlink{tocsection.\thesection}{#1}}}

\newcommand{\conteudo}{\tableofcontents\label{tocsection}}


\pagestyle{fancy}


\fancyhead[CO]{\title}
\fancyhead[CE]{\subtitle}
\fancyhead[R]{}
\fancyhead[L]{}
\fancyfoot[C]{\thepage}

\allowdisplaybreaks

\newif\ifdebug
\newcommand\todo[1]{\ifdebug {\color{red}#1}\fi}
\newcommand\doing[1]{\ifdebug {\color{blue}#1}\fi}