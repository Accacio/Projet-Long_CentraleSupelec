% Generated by GrindEQ Word-to-LaTeX 
\documentclass{article} %%% use \documentstyle for old LaTeX compilers

\usepackage[english]{babel} %%% 'french', 'german', 'spanish', 'danish', etc.
\usepackage{amssymb}
\usepackage{amsmath}
\usepackage{txfonts}
\usepackage{mathdots}
\usepackage[classicReIm]{kpfonts}
\usepackage[dvips]{graphicx} %%% use 'pdftex' instead of 'dvips' for PDF output

% You can include more LaTeX packages here 


\begin{document}

%\selectlanguage{english} %%% remove comment delimiter ('%') and select language if required


\noindent Trajectoires

\noindent 

Dans le cadre d'une t\^{a}che robotique, le mouvement de l'outil dans l'espace est d\'{e}fini par une s\'{e}quence de points travers\'{e}s dans temps. Cette s\'{e}quence et son historique temporel est d\'{e}sign\'{e} par \textbf{trajectoire}. Une t\^{a}che est g\'{e}n\'{e}ralement d\'{e}finie comme le mouvement entre une s\'{e}rie de points vis\'{e}s, par exemple � Aller de la position initiale au point A, s'approcher de la pi\`{e}ce jusqu'au point B, fermer l'outil en saisissant la pi\`{e}ce, aller au point C en portant la pi\`{e}ce~�, et l'objectif de la g\'{e}n\'{e}ration de trajectoire est de calculer le chemin travers\'{e} dans le temps parmi des points interm\'{e}diaires et ses respectifs temps. Une fois calcul\'{e}e, la trajectoire est mise comme r\'{e}f\'{e}rence \`{a} l'entr\'{e}e de la commande afin que l'effecteur suive le chemin souhait\'{e}.

La trajectoire peut \^{e}tre d\'{e}finie en termes de coordonn\'{e}s articulaires ou cart\'{e}siennes, o\`{u} la premi\`{e}re est mieux adapt\'{e} pour un chemin libre entre deux points et la deuxi\`{e}me pour un chemin contraint. Dans ce travail on a d\'{e}cid\'{e} de g\'{e}n\'{e}rer les trajectoires en coordonn\'{e}es cart\'{e}siennes, ce qui demande une commande en espace op\'{e}rationnel ou l'utilisation du mod\`{e}le g\'{e}om\'{e}trique inverse, pour les convertir en coordonn\'{e}es articulaires. 

A part les contraintes g\'{e}om\'{e}triques de la trajectoire, il y a les contraintes temporelles, c'est-\`{a}-dire, contraintes par rapport aux vitesses et acc\'{e}l\'{e}rations en chaque point de la trajectoires et dur\'{e}es maximales du mouvement. Un des principes de conception d'un g\'{e}n\'{e}rateur de trajectoire c'est l'utilisation de trajectoires lisses, une fois que physiquement c'est impossible de traverser l'espace de fa\c{c}on non-continue. En plus on utilise souvent courbes continues en vitesse et acc\'{e}l\'{e}ration afin de minimiser les soucis par rapport aux vibrations et r\'{e}sonances m\'{e}caniques. Afin d'am\'{e}liorer les vitesses, plusieurs g\'{e}n\'{e}rateurs de trajectoire utilisent techniques d'interpolation polynomiale et spline, mais dans le cadre de ce travail on ne les utilisera pas.

\noindent 

\noindent \textbf{Choix de conception et~Mise en {\OE}uvre}

\noindent 

 Inspir\'{e} par les choix de conception des robots Fanuc, on a d\'{e}cid\'{e} de mettre en {\oe}uvre un g\'{e}n\'{e}rateur de trajectoires avec trois fonctions basiques~: trajectoire lin\'{e}aire entre 2 points, trajectoire circulaire entre 2 points avec un point interm\'{e}diaire et trajectoire en arc de cercle entre 2 points avec un point interm\'{e}diaire. Avec ces trois types de trajectoire le robot sera capable de r\'{e}aliser toutes ses t\^{a}ches.

    

\noindent \textbf{Trajectoire Lin\'{e}aire}

 Le type plus simple de trajectoire, la trajectoire lin\'{e}aire consiste en aller d'un point initial ${\boldsymbol{P}}^{\boldsymbol{i}}$${}^{ }$au point final ${\boldsymbol{P}}^{\boldsymbol{f}}$\textbf{ }en suivant une ligne droite, o\`{u} tous les deux points sont vecteurs tridimensionnels et appartiennent \`{a} l'espace op\'{e}rationnel du robot. Soit $\boldsymbol{tf}$ le temps total du mouvement, la trajectoire peut \^{e}tre d\'{e}crite analytiquement par~:
\[\boldsymbol{P}\left(t\right)=\left({\boldsymbol{P}}^f{-\boldsymbol{P}}^i\right)\bullet r\left(t\right)+\ {\boldsymbol{P}}^i\ \ \ ,\ \ \ 0\ge t\ge tf\] 

O\`{u} r(t) est une fonction monotone continue du temps avec les conditions limites suivantes~:
\[r\left(0\right)= 0\] 
\[r\left(tf\right)=1\] 

On peut facilement observer que $\boldsymbol{P}\left(0\right)=\ {\boldsymbol{P}}^{\boldsymbol{i}}\ $et que$\ \boldsymbol{P}\left(tf\right)=\ {\boldsymbol{P}}^{\boldsymbol{f}}$, alors la contrainte g\'{e}om\'{e}trique est satisfaite. Pour satisfaire les contraintes de temps on d\'{e}finira $r\left(t\right)$ comme un polyn\^{o}me d'interpolation de 5${}^{\`{e}me}$ d\'{e}gr\'{e}e avec les caract\'{e}ristiques suivantes~:
\[\dot{r}\left(0\right)= 0\] 
\[\dot{r}\left(tf\right)= 0\] 
\[\ddot{r}\left(0\right)= 0\] 
\[\ddot{r}\left(tf\right)= 0\] 
Avec ces 6 conditions limites le polyn\^{o}me est bien d\'{e}fini~:
\[r\left(t\right)=10\bullet {\left(\frac{t}{tf}\right)}^3-15\bullet {\left(\frac{t}{tf}\right)}^4+6\bullet {\left(\frac{t}{tf}\right)}^5\] 
La figure suivant montre l'allure du polyn\^{o}me d'interpolation r(t), bien comme ses d\'{e}riv\'{e}es.

\noindent \includegraphics*[width=6.00in, height=4.50in, keepaspectratio=false]{image1}

\noindent \textit{Figure 1: r(t) e ses d\'{e}riv\'{e}es}

\noindent La trajectoire dans l'espace est montr\'{e}e dans la figure suivante. Les points sont espac\'{e}s d'un m\^{e}me intervalle de temps afin d'indiquer l'effet de la vitesse sur la trajectoire.

\noindent \includegraphics*[width=6.48in, height=4.86in, keepaspectratio=false]{image2}

\noindent \textit{Figure 2: Trajectoire lin\'{e}aire}

\noindent 

\noindent \textbf{Trajectoire Circulaire}

\noindent \textbf{}

\noindent s

\noindent 

\noindent 


\end{document}

