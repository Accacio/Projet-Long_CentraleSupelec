\mysubsection{Communication avec le robot}

A fin de établir la communication entre l'ordinateur et le robot on a choisi d'utiliser l'outil ROS pour faciliter cette communication. ROS est un système d'exploitation open source que a compatibilité avec diverses robots. Plus de détailles sur les résultats, comment les tests ont été faites en \ref{resultats/commum}  et quelques tutoriels en \ref{appen:commum}.

\mysubsubsection{Études initiaux}
Pour meilleur comprendre le paradigme et fonctionnement du ROS nous avons étudier toute la structure des n\oe{}uds (nodes) , services, messages, topiques, publicateurs (publishers) et abonnés (subscribers). Nous avons utilisés la documentation du ROS \cite{rosdoc} pour ces études. 

\mysubsubsection{Premiers Tests}
Comme les premiers tests, nous avons utilisé les tutoriels des sites \cite{matlabrobotics} et \cite{rosdoc} pour apprendre comment utiliser le robot turtle\_bot. Après nous avons fait tests en utilisant les tutoriels du turtle\_bot en \cite{matlabrobotics}, première utilisant le bash, après bash et matlab, et après, finalement utilisant le  bash et simulink.

\mysubsubsection{Tests du you\_Bot}
Après les tests basiques, nous avons commencé a réaliser les tests utilisant le bash, après bash et matlab, et finalement bash et simulink. 
