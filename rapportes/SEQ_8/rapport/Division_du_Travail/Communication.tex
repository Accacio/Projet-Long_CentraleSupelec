
\mysubsection{Communication avec le robot}

Cette partie concerne l’intégration entre le programme de commande et l’interface du robot Kuka youbot. Donc la première tâche à être accomplie c’est l’étude de cette interface de sorte qu’on puisse définir des tâches et planifier les étapes de cette phase.

\begin{itemize}
\item 2.A.0 Étude de l’interface et planification

\end{itemize}

Après l'étude géral du problème nous avons divisé en diverses tâches:

\begin{itemize}
	\item 2.A.1 Étude du ROS : 3 Crénaux;
	\item 2.A.2	Tests turtle\_bot utilisant bash/ROS : 3 Crénaux;
	\item 2.A.3	Tests turtle\_bot utilisant bash/ROS/Matlab : 3 Crénaux;
	\item 2.A.4	Tests turtle\_bot utilisant bash/Simulink : 3 Crénaux;
	\item 2.A.5	Tests you\_Bot utilisant bash/ROS : 3 Crénaux;
	\item 2.A.6	Tests you\_Bot utilisant bash/ROS/Matlab : 3 Crénaux;
	\item 2.A.7	Tests you\_Bot utilisant bash/ROS/Simulink : 3 Crénaux;
\end{itemize}