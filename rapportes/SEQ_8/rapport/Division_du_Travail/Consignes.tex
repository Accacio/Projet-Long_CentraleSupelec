\mysubsection{Commande Angulaire et Cartésienne}

Cette partie du projet a été la première où on a travaillé, car pour commencer un projet on doit premièrement étudier le sujet. Cela correspond aussi à la première étape du développement d'une loi de commande: étudier et modéliser le système. Par contre, le modèle du robot nous a été fourni dans un fichier Matlab et donc on a employé du temps pour le comprendre. 

L'étape initiale d'étude des systèmes robotiques a été la première tâche de tous les membres de l'équipe, une fois que le but du projet est d'apprendre comment les systèmes robotiques fonctionnent et comment les contrôler. De plus, on avait besoin de connaître les spécificités de ce type de système et tous les outils qu'on avait disponibles pour définir une application final qui était faisable dans le temps disponible. 

On a réservé initialement 28 créneaux pour les tâches rélationée à la commande du robot. Par contre, des problèmes de performance des lois de commande nous ont emmenés à consacrer plus de temps dans cette partie du projet. Dans la section \ref{Suivi} on a le calendrier du déroulement réel du projet. 

Dans la suite, la division de cette partie du travail en tâches est décrite, ainsi que la quantité de créneaux qu'on avait l'intention de consacrer à chaque tâche.

\begin{itemize}
\item 1.B.1: Étude à propos de la modélisation des systèmes robotiques: 6 créneaux;

\item 1.B.2: Compréhension du modèle Matlab du robot fourni: 2 créneaux

\item 1.B.3: Étude et développement des techniques de commande du robot dans l'espace des articulations et choix d'une méthode de commande: 4 créneaux;

\item 1.B.4: Mis en oeuvre de la loi de commande du robot dans l'espace des articulations dans le Simulink et tests: 2 créneaux;

\item 1.B.5: Étude et développement des techniques de commande du robot dans l'espace opérationnel et choix d'une méthode de commande: 2 créneaux;

\item 1.B.6: Mis en \oe{}uvre de la loi de commande du robot dans l'espace opérationnel dans le Simulink et tests: 4 créneaux;

\item 1.B.7: Optimisation des lois de commande: 2 créneaux;

\item 1.B.8: Intégration avec le module de génération des trajectoires: 3 créneaux;

\item 1.B.9: Intégration avec le robot: 3 créneaux;

\end{itemize} 
