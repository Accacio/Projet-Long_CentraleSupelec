\mysubsection{Interface Homme $ \leftrightarrow $ Machine }

Comme point de départ du développement de l’IHM, on a l’obtention et l’utilisation d'une fonction concise obtenue en tant que produit final de l’intégration effectuée à la fin de la phase 1 (tâche 1.B.8).

Dans un premier temps, on doit définir les caractéristiques, les contraintes et  les dimensions de l’environnement de travail de l’outil (tableau du jeu). Alors, on doit créer les fonctions pour exécuter les deux mouvements que le robot réalisera: “X” et “O”. Après, on doit définir les paramètres d’entrée et les données nécessaires pour le développement de l’interface et aussi la création et l’administration des objets graphiques de l’interface. Pour finir, on a besoin de faire des tests, des corrections nécessaires et des améliorations dans l’interface et l’intégration de ses fonctionnalités avec d'autres produits finaux du projet. 

Dans la suite, on a ordonné ces tâches et le temps estimé pour l'exécution de chaque activité:
\begin{itemize}

\item	2.B.0: Fonction concise qui détermine le paramètre d’entrée du robot (couple) à partir de la position finale et le type de trajectoire désirée (obtenue dans la tâche 1.B.8): 6 créneaux; 
\item	2.B.1: Définition de les caractéristiques et dimensions d’environnement de travail (tableau du jeu): 2 créneaux;
\item	2.B.2: Définition et création de la structure de données et paramètres nécessaires pour l’IHM: 2 créneaux;
\item	2.B.3: Création des fonctions de dessin “X” et “O”: 2 créneaux;
\item	2.B.4: Définition de la disposition graphique et des objets de l’interface: 2 créneaux;
\item	2.B.5: L’insertion des éléments dans l’interface: 2 créneaux;
\item	2.B.6: Création et gestion des événements associés à les objets: 2 créneaux;
\item	2.B.7: Tests, dépannage et intégration: 3 créneaux

\end{itemize}

On prévoit 6 créneaux pour la réalisation de la tâche 2.B.0 qu'on doit obtenir comme résultat de l’intégration des activités dans la première phase du projet et plus 15 créneaux pour l’exécution des autres tâches de l'interface.
\pagebreak
