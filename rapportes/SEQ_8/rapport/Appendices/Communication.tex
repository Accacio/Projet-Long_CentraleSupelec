\section{Tutoriels you\_Bot / Matlab}
\label{appen:commum}
\subsection{Installation you\_Bot}
Pour installer le pilote you\_Bot, il faut premièrement avoir ROS installé dans la machine, dans notre cas nous avons utilisés une machine virtuel avec ROS Indigo pré-installé. La machine est disponible utilisant le lien en \cite{downloadmachine}.\\
Aprés, il faut courir les commandes au terminal:


\begin{verbatim}
	sudo apt-get install ros-indigo-youbot-driver
	sudo ldconfig /opt/ros/indigo/lib
	sudo apt-get install ros-indigo-youbot-driver-ros-interface \
	    ros-indigo-youbot-description
	sudo setcap cap_net_raw+ep \
	    /opt/ros/indigo/lib/youbot_driver_ros_interface\
	    /youbot_driver_ros_interface
	sudo ldconfig /opt/ros/indigo/lib
\end{verbatim}
	Maintenant le pilote et le \flqq{}wrapper\frqq{} sont installés. Pour installer le simulateur gazebo et les modèles du you\_Bot il faut courir:

\begin{verbatim}	
	sudo apt-get install ros-indigo-ros-control ros-indigo-ros-controllers\
	    ros-indigo-gazebo-ros-control
	cd <your-catkin-folder>/src
	git clone http://github.com/youbot/youbot_description.git -b indigo-devel
	git clone http://github.com/youbot/youbot_simulation.git
	catkin_make
	source <your-catkin-folder>/devel/setup.bash
\end{verbatim}

Où \verb|<your-catkin-folder>| est le nom du \flqq catkin workspace\frqq{} utilisé, \cite{roscatkin}.
Pour tester il faut courir:

\begin{verbatim}
roslaunch youbot_gazebo_robot youbot.launch 
\end{verbatim}



Et dans autre terminal:

\begin{verbatim}
rosrun youbot_driver_ros_interface youbot_keyboard_teleop.py
\end{verbatim}

Maintenant utilisant les commandes donnés par le programme, on peut mouvementer le robot. Pour faire le test des articulations il faut courir le code suivant au lieu du antérieur:

\begin{verbatim}
	rosrun youbot_driver_ros_interface youbot_arm_test
\end{verbatim} 
		
Chacun de ces programmes peuvent être arrêtés par \verb|^C|.


\paragraph{Remarque:} Le code \verb|roslaunch youbot_gazebo_robot youbot.launch| charge dans le simulateur le modèle du robot avec sa base roulante, au lieu de ce code il faut utiliser le code suivant, pour le modèle sans la base :

\begin{verbatim}	
roslaunch youbot_gazebo_robot youbot_arm_only.launch
\end{verbatim}

\subsection{Configuration Matlab/ROS/you\_Bot}
Avec une distribution récent du Matlab (R2015a par exemple) et que aie le \flqq{}Robotics System Toolbox\frqq{} installé ( lien et tutoriels en \cite{matlabrobotics} ). Il faut le configurer, pour commencer un node ROS au matlab et le connecter a un master (dans nos cas, le robot ) il faut courir cet code:
\begin{verbatim}
rosinit('host_ip')
\end{verbatim}

Où \verb|host_ip| est l'adresse ip du master, 192.168.1.1 par exemple. Pour s'assurer de la valeur du adresse ip, il faut courir dans la machine où le robot est connecté:

\begin{verbatim}
ifconfig
\end{verbatim}

Et dans le terminal il aura la valeur du ip de la machine. Cas il n'y est pas là, consulter la connexion de la machine avec le réseaux.

Après la connexion entre Matlab et la machine connecté au robot, il faut le tester, donc on peut créer un abonné a partir du matlab pour lire les informations données par le robot:


\begin{verbatim}
joints_subs = rossubscriber('/joint_states')
\end{verbatim} 

Et après pour recevoir les valeurs:

\begin{verbatim}
joint_data = receive(joints_subs,10)
\end{verbatim} 

Et maintenant, nous avons les valeurs momentanées de position, vitesse et effort, dans nos cas l'angle, la vitesse angulaire et le couple en chaque joint.

\paragraph{Remarque:} Pour envoyer les couples il faut plus des temps d'étude, a fin de déterminer que topique a utiliser.

\subsection{Configuration Simulink/ROS/you\_Bot}

La connexion entre Simulink et la Machine est un peu similaire au qu'a été fait antérieurement mais purement graphique utilisant la \flqq{}GUI\frqq{} du Simulink.

Dans une nouvelle fenêtre du Simulink, et après sélectionner dans le menu \verb|Tools > Robot Operating System > Configure Network Addresses|, après saisir l'adresse ip du master, dans le champ \verb|Hostname/IP Adress| et cliquer en Ok.
 
Dans le modèle insérer un bloc \flqq{}Subscribe\frqq{}  , un bloc \flqq{}Terminator\frqq{}, un \flqq{}Bus Selector\frqq{} et 3 \flqq{}Scopes\frqq{}. 
 
Sélectionner le bloc \flqq{}Subscribe\frqq{} et le configurer pour recevoir le topique \verb|joint_states|.

Lier \verb|IsNew| dans le  terminal du bloc \flqq{}Terminator\frqq{} et \verb|Msg| dans la entrée du \flqq{}Bus Selector\frqq{}.

Dans \flqq{}Bus Selector\frqq{}, sélectionner Position, Vitesse et Effort.

Lier chacune sortie du \flqq{}Bus Selector\frqq{} a un \flqq{}Scope\frqq{} différent.

Après, faire courir la simulation. Chaque \flqq{}Scopes\frqq{} aura 7 courbes, chacune caractérisant les 7 articulations, 5 joint plus 2 acteurs des pinces.

\paragraph{Remarque:} Pour envoyer les couples il faut plus des temps d'étude, a fin de déterminer que topique a utiliser.