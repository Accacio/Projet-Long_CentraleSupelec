\documentclass[11pt,a4paper,onecolumn]{articlewithlogo}
\usepackage[utf8]{inputenc}
\usepackage[francais]{babel}
\usepackage[T1]{fontenc}
\usepackage{amsmath}
\usepackage{amsfonts}
\usepackage{amssymb}
\usepackage{color}
\usepackage{float}
\usepackage{graphicx}
\graphicspath{{Figures/}}
\usepackage{blindtext}
\newif\ifdebug
\newcommand\todo[1]{\ifdebug {\color{red}#1}\fi}
\debugfalse

\logo{logos/supelec.jpeg}{50mm}
\title{Rapport d'activité de la Semaine 1}
\author{Karoline CARVALHO BÜRGER\\ Tiago DE JESUS RODRIGUES\\  Rafael ELLER CRUZ \\ Clément HENNEUSE\\ Rafael Accácio NOGUEIRA }
\teacher{Mme. MAKAROV}
\date{\today}

\begin{document}
\maketitle
\section{Activités Réalisés le mercredi 28/09/16  }
\begin{itemize}
	\renewcommand\labelitemi{$\circ$}
	\item Familiarisation avec le logiciel Roboguide et son interface
	\item Création d'une nouvelle cellule et ouverture d'une cellule déjà programmée
	\item Observation d'une cellule programmée pour comprendre la structure de programmation 
	\item Compréhension des types de variable de positon
	\item Compréhension des différentes types de repère (world, tool, user)
	\item Lecture de l'avant-projet des  olympiades  de l'année dernière 
	\item Définition manuel d'un repère utilisateur
	\item Réalisation des quelques tutoriels et exercices
	d'introduction
	\item Utilisation du TEACH PENDANT pour faire la programmation du robot
\end{itemize}
\bibliographystyle{plain}
%\bibliography{bibliografia}
\end{document}